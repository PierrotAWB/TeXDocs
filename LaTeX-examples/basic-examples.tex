\documentclass[11pt]{article}

\def\macroText{This text was generated using a macro. \\}

\usepackage[margin=1.5in, paperwidth=8.5in, paperheight=11in]{geometry}
\usepackage{amsfonts}
\usepackage{graphicx}

\begin{document}

\title{Exploring \LaTeX \ Notation}
\author{Andrew}
\date{\today}
\maketitle

\newpage
\tableofcontents
\clearpage

\section{Math Modes}

Text wrapped between two single dollar signs are in in-line math made. Text will appear in the same line as preceding text. E.g., the side lengths of a rectangle can be represented by $(x+1)$ and $(x+3)$.  \\

Text wrapped between two double dollar signs are in display math mode. Text will appear on the next line. \\

The area, A, of the rectangle can be represented as: $$A=(x+1)(x+3)=x^2+4x+3$$

\section{Superscripts}

Superscripts use a caret symbol. 

$$2x^3$$
$$2x^34$$
$$2x^{34}$$

When the superscript is longer than one character, curly braces are used to group, or text wrap. 

$$2x^{3x^{5+3x}+7}$$

\section{Subscripts}

Subscripts use an underscore. 

$$x_1$$
$${x_1}_2$$
$${{x_1}_2}_3$$

When the subscript is longer than one character, curly braces are used to group, or text wrap.

\section{Greek Letters}

Greek letters are generated by a backslash and the letter spelt out in English. 

$$\pi$$
$$\alpha$$
$$\delta$$
$$A=\pi r^2$$

\section{Trigonometry}

Trigonometric functions are generated by a backslash and the abbreviated name. The argument is placed in curly braces.

$$y=\sin{x}$$
$$\arctan{\pi}$$

\section{Log Functions}

Logarithmic functions are generated by a backslash and the abbreviated name. The argument is placed in curly braces.

$$\log{x}$$
$$\ln{x}$$
$$\log_n{x}$$

To give the logarithm a base, use an underscore to produce the subscript (see above).

\section{Roots}

\subsection{Square Roots}

Square roots are generated by \texttt{\textbackslash sqrt}. The argument is placed in curly brackets that follow the keyword.

$$\sqrt{2}$$
$$\sqrt{x}$$

\subsection{Nth Roots}

A root with a degree other than 2 requires an additional argument. This additional argument is placed in square brackets that follow \texttt{\textbackslash sqrt} and precede the curly braces.

$$\sqrt[3]{2}$$
$$\sqrt[3]{{x^2}_\alpha+{y^n}_\beta}$$ \\

\subsection{Nested Roots}

Nested roots can be drawn by calling the root function within the curly braces of another root function.

$$\sqrt{1+\sqrt{x}}$$

\section{Fractions}

An in line fraction is generated by \texttt{\textbackslash frac}. After the keyword, two arguments are required: a numerator, and a denominator. Both are placed in curly braces. $\frac{2}{3}$ 

Sometimes the fraction may be too small. This is changed by using the keyword \texttt{\textbackslash displaystyle\{\}}. Place the fraction function call inside the curly braces of the \texttt{\textbackslash displaystyle\{\}} function to increase its size to that of normal font. $\displaystyle{\frac{2}{3}}$.

In display math mode, fractions do not shrink by default.

$$\frac{x}{x^2+x+1}$$ \\
$$\frac{\sqrt[3]{{x^2}_\alpha + {y^2}_\beta}}{\sqrt[3]{{x^2}_\alpha - {y^2}_\beta}}$$ \\
$$\frac{x}{1+\frac{1}{x}}$$

\section{Brackets}

Square brackets and parentheses are simply generated with [] and (), respectively. 

$$3(x+2)$$
$$3[2(x+1)]$$

By default, LaTeX will hide curly braces. To be visible, curly braces (both open and close) must be preceded by a backslash. 

$$\{a,b,c ... n\}$$

The same applies for dollar signs. 

$$\$ 12.50$$

Notice that a space must follow the dollar sign to be outputted. This is because if the characters after the dollar sign are attached to the dollar sign, the compiler cannot identify the keyword being called. \\

Applying brackets properly to fraction requires two new keywords:
\texttt{left} and \texttt{right}, both preceded by backslashes. The two must be used together in order to compile correctly. They are placed immediately before the bracket. \\

Without the keywords: $$6(\frac{x}{3})$$ \\

Notice that the brackets do not cover fully the fraction from top to bottom.\\

With the keywords: $$6\left(\frac{x}{3}\right)$$
$$6\left[\frac{x}{3}\right]$$
$$6\left\{\frac{x}{3}\right\}$$

Absolute value bars are generated simply with a pipe "$|$".

$$|x|$$
$$|\frac{x}{x+1}|$$

Note that the shortened pipes do not fully cover the fraction from top to bottom. This is, again, changed by using the \texttt{left} and \texttt{right} keywords.

$$\left|\frac{x}{x+1}\right|$$

Since the two keywords must both be present in order to compile correctly, there is a method to display only one of the two brackets. By adding a period (instead of the bracket) after one of the keywords, that side will be hidden. Consider the following:

$$\left.\frac{x^2}{6}\right\}$$
$$\left\{\frac{x^2}{6}\right.$$
$$\left.\frac{d}{dy} x\right|_{x=1}$$

\section{Tables}

Tables are generated by using the \texttt{begin} keyword. This is the same keyword used to create the document, only this time, the argument is not \texttt{document}, but rather \texttt{tabular}. Also, once it is established that a tabular is being called in "begin", another argument is required. This argument specifies the number of columns to appear in the table. For each column, type a \texttt{"c"}. \\

Entries in the table are typed in text mode and separated by ampersands. Ensure that the correct amount of entries appears in one row (the same as the number of columns). A row is ended by a double backslash. \\

\begin{tabular}{cccccc}
x & 1 & 2 & 3 & 4 & 5 \\
f(x) & 10 & 11 & 12 & 13 & 14 \\
\end{tabular} \\

To add a horizontal line between rows, the keyword "hline" is used after the double backslash ending the row. \\

\begin{tabular}{cccccc}
x & 1 & 2 & 3 & 4 & 5 \\ \hline
f(x) & 10 & 11 & 12 & 13 & 14 \\
\end{tabular} \\

To add vertical lines, pipes are placed next to the c's in the second argument of the begin function. \\

\begin{tabular}{c|c|c|c|c|c}
x & 1 & 2 & 3 & 4 & 5 \\ \hline
f(x) & 10 & 11 & 12 & 13 & 14 \\
\end{tabular} \\

Using these procedures, a fully enclosed table can be generated. \\

\begin{tabular}{|c|c|c|c|c|c|}
\hline
x & 1 & 2 & 3 & 4 & 5 \\ \hline
$f(x)$ & 10 & 11 & 12 & 13 & 14 \\ \hline
\end{tabular} \\

\section{Equation Arrays}

Equation arrays are useful for lining equations up by their equal signs. They are generated with the begin command by entering the argument \texttt{eqnarray}

\begin{eqnarray}
5x^2 - 9 = x^2 + 3 \\ 
4x^2 = 12 \\
x^2 = 3 \\
x\approx\pm{1.732}
\end{eqnarray} \\

Sandwiching each equation with ampersands lines the equations up by the equal signs.

\begin{eqnarray}
5x^2 - 9 &=& x^2 + 3 \\ 
4x^2 &=& 12 \\
x^2 &=& 3 \\
x&\approx&\pm{1.732}
\end{eqnarray} \\

To remove the equation numbers, replace the keyword (in the \texttt{begin} line and the \texttt{end} line) \texttt{eqnarray} with \texttt{enarray*}

\begin{eqnarray*}
5x^2 - 9 &=& x^2 + 3 \\ 
4x^2 &=& 12 \\
x^2 &=& 3 \\
x&\approx&\pm{1.732}
\end{eqnarray*}

\section{Lists}

Lists come in two forms: numbered lists, and bulleted lists. Using the keywords is preferred to manually creating the lists, as LaTeX numbers the items automatically. 

\subsection{Numbered Lists}
Numbered lists use the keyword \texttt{\textbackslash begin\{\}} with the argument \texttt{enumerate}. All items in the list are preceded by the keyword \texttt{\textbackslash item}.

\begin{enumerate}
	\item Pencil
	\item Calculator
	\item Graph paper
	\item Notebooks
	\item Highlighters
\end{enumerate}

\subsection{Bulleted Lists}
Bulleted lists use \texttt{\textbackslash begin\{\}} with the argument \texttt{itemize}. All items in the list are preceded by the keyword \texttt{\textbackslash item}.

\begin{itemize}
	\item Pencil
	\item Calculator
	\item Graph paper
	\item Notebooks
	\item Highlighters
\end{itemize}

\subsection{Nested Lists}
To increase levels on a list, simply begin another list after the item that requires subdivisions. 
\begin{enumerate}
	\item Pencil
	\item Calculator
	\item Graph paper
	\item Notebooks
		\begin{enumerate}
			\item Assessments
			\item Quizzes
			\item Tests
		\end{enumerate}
	\item Highlighters
\end{enumerate}
\dotfill
\begin{itemize}
	\item Pencil
	\item Calculator
	\item Graph paper
	\item Notebooks
		\begin{enumerate}
			\item Assessments
			\item Quizzes
			\item Tests
		\end{enumerate}
	\item Highlighters
\end{itemize}

\subsection{Customized Label Names}
Sometimes it makes more sense to use words instead of numbers or bullets. This is achieved by placing the desired label name within square brackets that immediately follow the \texttt{\textbackslash item} keyword. 

\begin{itemize}
	\item[Commutative] $a+b=b+a$
	\item[Associative] $(a+b)+c=a+(b+c)$
	\item[Distributive] $a(b+c)=ab+ac$
\end{itemize}

Notice that the item names are right justified. This may be reformatted. 

\section{Text Formatting}
LaTeX has different elements for different types of emphasis. Some change the overall outlook of the font, while others change only the size. Some control positioning. Consider the following examples: 

\begin{itemize}
	\item[Italics] This is \textit{italicized} font.
	\item[Bold-faced] This is \textbf{bold-faced} font.
	\item[Typewriter] This is \texttt{typewriter} font.
	\item[Small caps] This is \textsc{small caps} font.
\end{itemize}

These four text commands are generated by keywords that take the output text as input. For example, italicized font uses the keyword \texttt{\textbackslash textit} which is followed by curly braces containing whatever the desired output is. 

\begin{itemize}
	\item Bold-faced uses \texttt{\textbackslash textbf}.
	\item Typewriter uses \texttt{\textbackslash texttt}.
	\item Small caps uses \texttt{\textbackslash textsc}.	
\end{itemize}

As always, each keyword is preceded by a backslash. \\
Sometimes we wish only to change the size of the font. LaTeX uses the following six keywords for changing font size: 
\\

Normal font: The quick brown fox jumps over the lazy dog.

\begin{itemize}
	\item[\texttt{large} |] 
		\begin{large}
			The quick brown fox jumps over the lazy dog.
		\end{large}
	\item[\texttt{Large} |]
		\begin{Large}
			The quick brown fox jumps over the lazy dog.
		\end{Large}
	\item[\texttt{huge} |]
		\begin{huge}
			The quick brown fox jumps over the lazy dog.
		\end{huge}
	\item[\texttt{Huge} |]
		\begin{Huge}
			The quick brown fox jumps over the lazy dog.
		\end{Huge}
	\item[\texttt{small} |]
		\begin{small}
			The quick brown fox jumps over the lazy dog.
		\end{small}
	\item[\texttt{tiny} |]
		\begin{tiny}
			The quick brown fox jumps over the lazy dog.
		\end{tiny}
\end{itemize}

To centre, left-justify, or right-justify, the \texttt{\textbackslash begin\{\}} keyword is used. To centre, use the argument \texttt{center}. To left-justify, use the argument \texttt{flushleft}. To right-justify, use the argument \texttt{flushright}.

\begin{center}
	This is centred.
\end{center}

\begin{flushleft}
	This is left-justified.
\end{flushleft}

\begin{flushright}
	This is right-justified.
\end{flushright}

\subsection{Special Formatting}

The word "LaTeX" (case-sensitive), when preceded by a backslash, will display in a different style. 

\LaTeX \\

To add in blank spaces, type a backslash followed by a blank space.
\\
Hello,\ \ \ \ \ world! \\

Notice that repeatedly typing spaces does not achieve this effect.
\\
Hello,          world! \\

\section{Packages}

The standard LaTeX library is not all inclusive. Other "packages" can be downloaded for additional content and control. For example, the geometry package allows users to specify paper size and margin size.\\ 
The \texttt{\textbackslash usepackage\{\}} keyword is used in the preamble after the \texttt{\textbackslash documentclass[]\{\}}. It takes an argument in curly braces that specifies the desired package. \\
Here are some examples of packages:

\begin{itemize}
	\item geometry
	\item fullpage
	\item amsfonts
	\item graphicx
\end{itemize}

Here is the amsfonts package at play: \\

The set of Natural numbers is denoted by $\mathbb{N}$

The set of Real numbers is denoted by $\mathbb{R}$

The set of Complex numbers is denoted by $\mathbb{C}$.

\section{Macros}

"Macros" are user defined commands, which can be used in LaTeX documents. Macros are defined in the preamble (where the \texttt{\textbackslash documentclass[]\{\}} and \texttt{\textbackslash usepackage\{\}} keywords are used) using the keyword \texttt{\textbackslash def}. Seeing as it is a keyword, \texttt{\textbackslash def} must be preceded by a backslash. The name of the macro follows \texttt{\textbackslash def} and is also preceded by a backslash. Finally, the curly braces are placed and are filled with code that may be called upon by the name of the macro at a later time. \\

Calling the macro works the same way as it would with standard functions. Type a backslash and follow with the name of the function. Note that no argument is needed for the macro.\\

\macroText 
\macroText
\macroText
\macroText
\macroText \\
If you are defining a macro as an equation, be sure to use math mode to call it. This can be done by defining the macro with dollar signs (preferable), or calling the macro in math mode.

\section{Graphics}

\begin{figure}[h]
	\includegraphics[scale=0.25]{logo.png}
\end{figure}

Graphics are created using the \texttt{graphicx} package.
The function \texttt{includegraphics[]\{\}} has an optional argument specifying image size in square brackets as well as the file name. Pdf, Png, Jpg, and Jpeg are all acceptable file types.

The square bracket argument generally uses the word \texttt{scale}. For example, we might write "scale=0.5", which scales the image size to 50\% its original value. Alternatively, we might use a more absolute measurement. \texttt{width} is used like so: "width=5in". The width may be specified with inches "in", millimetres "mm", or centimetres "cm". 

Further, the angle of the image can be changed. By using the word \texttt{angle} and setting it equal to some value in degrees, the image will rotate counter-clockwise about the bottom left corner the specified angle. 

\subsection{Text-wrap Formatting}
	To caption our figures, we must wrap the \texttt{\textbackslash includegraphics} function with \texttt{\textbackslash begin\{figure\}} and \texttt{\textbackslash end\{figure\}}. After the \texttt{\textbackslash begin\{figure\}}, type a pair of square brackets that contain specifications to text wrapping options. 

\begin{itemize}
	\item \texttt{h} holds the order of the text and the image. For example, if in the code, a line of text is followed by the image, using \texttt{h} will display it in that order. 
	\item \texttt{t} moves the image to the top.
	\item \texttt{b} moves the image to the bottom. 
	\item omitting this argument allows LaTeX to automatically wrap text.
\end{itemize}

Using one of these arguments looks like this: \texttt{\textbackslash begin\{figure\}[h]}

\subsection{Captioning}
For basic captioning, use the \texttt{caption} package. Place the keyword \texttt{\textbackslash caption\{\}} between \texttt{\textbackslash begin\{figure\}} and \texttt{\textbackslash end\{figure\}}. Type, within the curly braces, the desired caption. For example, \texttt{\textbackslash caption\{Figure 1. The canis lupis.\}}
%blaze it%

\begin{thebibliography}{1}

\bibitem{lamport94}
  Leslie Lamport,
  \emph{\LaTeX: a document preparation system},
  Addison Wesley, Massachusetts,
  2nd edition,
  1994.

\end{thebibliography}

%$$2\left( \sum^{\infty}_{n=1}{\frac{1}{2^n}}\right)$$

\end{document}


