\documentclass[12pt]{article}
\usepackage{indentfirst, setspace, fancyhdr, hyperref, xcolor, textcmds}

\definecolor{jade}{RGB}{61, 122, 122}

\pagestyle{fancy}
\rhead{Introduction to the DECA Case Study}
\lhead{}

\hypersetup{
	colorlinks = true,
	linkbordercolor = jade,
	linkcolor = jade,
	pdfborderstyle={/S/U/W 1},
	urlcolor = jade
}

\makeatletter
\Hy@AtBeginDocument{%
  \def\@pdfborder{0 0 3}% Overrides border definition set with colorlinks=true
  \def\@pdfborderstyle{/S/U/W 1}% Overrides border style set with colorlinks=true
                                % Hyperlink border style will be underline of width 1pt
}
\makeatother

\makeatletter
\newcommand*{\toccontents}{\@starttoc{toc}}
\makeatother

\begin{document}\thispagestyle{empty}

% TITLE

	\newcommand{\HRule}{\rule{\linewidth}{0.5mm}} 
	\begin{center}

	% \vspace*{50px}
	\HRule\\[0.4cm]
	
	{\huge\bfseries An Introduction to the DECA \textit{Case Study}}\\[0.4cm] 
	
	\HRule\\[0.4cm]
	{Written by: \textsc{Andrew Wang}} \\[1cm]
	{\large\bfseries \textsc{CONTENTS}}\\[0.4cm] 
	\vspace*{.5\baselineskip}
	\toccontents
	\end{center}
	
\vspace*{3\baselineskip}

\textbf{Preface.} Before reading this document (or putting it off), please acknowledge two things. First, this document is largely aimed at those with little to no DECA experience, though my hope is that regardless of how much experience you have, you may learn something of usefulness. Second, the only thing that can substitute hard work is smart work. \textit{Understand} everything what you learn and \textit{master} it, but do not waste your time on the insignificant.

\section{What is a case study?}

\subsection{Overview}

	Depending on your event, a \textit{case study} can make up either half (team events) of your composite DECA score, or as much as two thirds (individual events). It is a quintessential DECA activity wherein participants are expected to both demonstrate and apply their industry knowledge to a particular scenario, hence the name. Participants may find their case studies akin to a \lq role play' type interaction with their judges. \\ 
	
	 At the end of the case, the judge will typically ask two or three questions that were given to them along with the case (the participant will not know beforehand what these questions entail). In some cases, the participant will have already answered the questions in his case study and the judge will not ask them. Other times (usually at the provincial level or higher), a judge may even choose to ask her own questions, independent of what they are required to ask. \\
	
	In order to succeed in a case study, it is often necessary not only to solve a problem, but to justify why your approach is both efficient and effective. Other times, you might be asked to develop and pitch a policy ideas to a senior officer. It may even be something outside your typical 'business' context; there have been case studies in which candidates were asked to make a presentation (inside the presentation) in order to inform, say, parents, or students (or some other \lq\textit{non-technical}' audience). Nonetheless, one thing is common: you must present both \textbf{confidently} and \textbf{informatively}. \\
	
	If you have historically not done so well, or have not done a case study at all, know that ,as with most things, good presentation (specifically, case study) skills can be learned. Even if you are faced with a scenario of which you know very little, \lq \textit{faking it}' will be better than doing nothing. Incorporate what you \textit{do} know with confidence.

\subsection{Logistical Details}

	Individual events, which in the finance 'cluster' are: PFL, ACT, and BFS, give participants 10 minutes to prepare for a 10 minute (maximum) case. Team events, namely FTDM, giver 30 minutes to prepare for a 15 minute (maximum) case. \\
	
	That is not a lot of time. It makes it virtually impossible to plan your presentation \textit{word by word} and still do well. Therefore, it is necessary to skip the finer details and focus on the bigger picture: i.e., to brainstorm ideas, but not \lq fleshed out' until they are presented. \\

	ACT and BFS will typically involve financial analysis (this may or may not involve numbers; you might be asked to do some calculations based on data they give you). Participants usually act as accountants or financial advisers, who, with their experience, knowledge, and professional judgement, are expected to provide a recommendation or conclusion to the inquiring business manager/officer. FTDM cases often have to do with banking; you and your partner will work together to perform, say, a bank reconciliation\footnote{See Investopedia's \href{http://www.investopedia.com/terms/b/bankreconciliation.asp}{Bank Reconciliation article}}. This is also typically for a business owner/officer. PFL cases deal with financial knowledge in the scope of individuals. You may be asked to give advice to, say, a university student, on the use of credit, or investment, or even insurance. \\

\section{Grading}

For an example case study, see the following \href{https://www.deca.org/wp-content/uploads/2014/08/HS_BFS_Web_Sample.pdf}{BFS case}. 

\subsection{Performance Indicators}

	Performance indicators (P.I.'s) act as a guideline for your presentation's content. They are points that  should be argued for or against, or concepts that are to be explained, in addition to anything you choose to add in the presentation. Often times, using the exact phrasing of the P.I. allows judges to pick up on which one you are discussing. For example, in the example case (page 1, P.I. \#1), the participant might say something along the lines of, \lq  Managerial accounting techniques are a crucial component to business management. It is through planning and budgeting that businesses can make informed decisions and minimize risk, and, at the same time, it allows us to keep track of the progress of our projects and gauge their success.'\footnote{Example adapted from Investopedia's \href{http://www.investopedia.com/ask/answers/062915/what-are-common-concepts-and-techniques-managerial-accounting.asp}{Managerial Accounting Techniques article}} It is desirable that, in a presentation with decent content, the judge finds it easy to follow the argument(s). \\

	At the regional competition, since judges are potentially less experienced and/or knowledgeable, it's often a good idea to structure your presentation \textbf{in the same order as the P.I.'s are given}. At the provincial level (though preferably higher) it  becomes practical to structure the presentation to flow nicely, i.e., not to skip around. Flow and structure of presentation is addressed in more detail in section 4. It is a good idea to spend a fairly even amount of time on each P.I. because if you have a highly structured presentation, it becomes easy to spot where you are skimping on explanation. \\
	
	Finally, although P.I.'s may govern what you \textit{ought} to do, they do not necessarily restrict what you \textit{can} do. As you'll see in section 4, there are many things you can do outside of or alongside your P.I.'s that may refine your presentation.
	
\subsection{$\textrm{21}^{\textrm{st}}$ Century Skills}

	Although historically, almost all of a case study's grade would be determined by the content rather than its delivery, it has become increasingly true that the communication of the participant(s)'s ideas is important. Enter the $\textrm{21}^{\textrm{st}}$ Century Skills that DECA has only recently introduced. \\
	
	These skills are now listed on the rubric (see pg. 9 of the example case), instead of the old criterion: \lq Overall impressions'. Essentially, the difference now is that the old \lq overall impressions' have been broken down into subcategories, though seemingly invariant at that. In particular, it seems that these skills are unchanging in each case study, and the will indefinitely serve only to judge creativity and general execution. Nonetheless, it might help to know that they are as follows: 
	
\begin{enumerate}
		\item{Reason effectively and use systems thinking?}
		\item{Make judgments and decisions, and
solve problems?}
		\item{Communicate clearly?}
		\item{Overall impression and responses to
the judge’s questions}
\end{enumerate} 
	
\subsection{Mark Schemes}

	PFL has 3 P.I.'s per case study, each valued at 24 points. 28 points are allocated to $\textrm{21}^{\textrm{st}}$  Century Skills. ACT and BFS both have 5 P.I.'s per case study, each valued at 14 points. A maximum of 30 points may be attained in the $\textrm{21}^{\textrm{st}}$ Century Skills section. FTDM has 7 P.I.'s per case study, each valued at 10 points. A maximum of 30 points may be attained in the $\textrm{21}^{\textrm{st}}$ Century Skills section. \\

\section{Resources}

	To familiarize yourself with the content of the content that may appear on a case study, please refer to the following lists of resources. You might observe that there is overlap between what one might study for her exam and her case study.
	
\subsubsection*{P.I.'s, Notes, and Archived Material}
	
	  \href{http://www.deca.ca/documents/Performance\%20Indicators/ACT.docx}{ACT}, 
	  \href{http://www.deca.ca/documents/Performance\%20Indicators/BFS.docx}{BFS}, 
	  \href{http://www.deca.ca/documents/Performance\%20Indicators/FTDM.docx}{FTDM}, and 
	  \href{http://www.deca.ca/documents/Performance\%20Indicators/PFL.docx}{PFL}.  					\href{ilcollaborate.org/wp-content/uploads/2014/11/Finance-Indicators-Explained.docx}{Finance Bell Ringer activities}, a sort of text. \href{https://saltfleetdeca.commons.hwdsb.on.ca/sample-page/}{Previous DECA cases}.

\subsubsection*{Guides to a Better Presentation}
\begin{itemize}
	\item[-] \href{https://issuu.com/decainc/docs/deca_competitive_events_success}{DECA's own \lq Tips and Tricks' pamphlet}
	\item[-] \href{https://www.deca.org/high-school-programs/competitive-events-sample-videos-hs/}{DECA's ICDC example case studies}, these are not in finance, but are instead exemplars in delivery, and for those with no DECA experience, a sort of \lq sneak peek'.
\end{itemize}

\subsubsection*{Content Knowledge}
	\href{http://www.investopedia.com/}{Investopedia}. The Khan Academy \href{https://www.youtube.com/playlist?list=PL83DF21B47327EDFE}{Personal Finance},  \href{https://www.youtube.com/playlist?list=PLXuBZowfrFgYddDQpXdDtViretdX4ECKS}{Accounting and Financial Statements}, and \href{https://www.youtube.com/playlist?list=PLCECDA315A8848B99}{Banking and Money} playlists. This \href{https://www.youtube.com/watch?v=mhmaHayMha8}{5 minute finance lesson}. This \href{https://www.youtube.com/watch?v=B7300KsDdYY}{overview of the three major financial statements}. 
	
	ACT participants may find it helpful to go over more \lq accounting specific' content: try videos 2-6, 11, 17-18, 38-39, and 58 of \href{https://www.youtube.com/playlist?list=PL_PmoCeUoNMIX3zP2yYSAq8gi6irBVh-1}{this playlist}.



\section{Tips and Tricks for Success}

If we compare the case study to a road trip, then the performance indicators are simply the stopovers. They do not necessarily restrict the medium of travel, the routes you take, and in a general (but almost vague) sense, \textit{how} you travel. In the same regard, a case study will need to address certain issues, but there is a lot of flexibility in and influence from the way the arguments are presented. This section will cover general presentation techniques, \lq secret weapons', and media of presentation.

As, with the general \lq good' presentation, there are many ways a participant may demonstrate that his oratorical skill. Here is a small list of habits, that, when put together, may enhance what would otherwise be a satisfactory presentation:

\begin{itemize}
	\item[-] \textbf{Be confident}
	\item[-] Be professional and courteous
	\item[-] Maintain fluency
	\item[-] Have A strong introduction
	\item[-] Exude enthusiasm, excitement, and fun
	\item[-] Maintain strong (but not \textit{too} strong eye contact
	\item[-] Maintain good posture (sit an inch or so from the edge of the seat)
	\item[-] Limit the use of crutch words, e.g., "um", or "uh"
	\item[-] Give a firm handshake
	\item[-] Speak at a reasonable volume (project your voice)
	\item[-] Create structure in your presentation
\end{itemize}

For FTDM cases, partners should show agreement with whatever the other person is saying. This can provide reassurance both to the speaker and the judge, and it shows overall maturity. For instance, one partner might nod his head, or say something like \lq as [partner's alias] said, [...]'.  In order to best synergize, divide the workload in half (roughly, since there are 7 P.I.'s) and agree upon an order in which to present. If you would like to send some sort of note to your partner (e.g., \lq let me speak', \lq hurry up', etc.) agree beforehand that a tap of her leg will have \textit{this} particular meaning (don't go overboard, or it'll become confusing). \\

Aside from the behavioural aspect of a case study, it is also possible to \lq practice' the content of the general case study. If you prepare three or more \lq secret weapons', e.g., acronyms, quotes, analogies, mottos, etc., that are applicable to a wide range of scenarios, it is then possible to have prepared (maybe even edited!) content that you may bring into your case, in accordance with the rules. Introductions and conclusions will also remain relatively similar to each other regardless of the scenario. Therefore, it is useful to practice introducing yourself and summarizing your points. Learn to give that firm handshake and greet the judge with a warm smile. You can even provide contact information (please, keep it appropriate and fake) to the judge along with the line: \lq Thank you, and remember that if you need to ask me anything, I'll always be available via email or phone.' \\

Along with the \textit{B.E.T.} example in point 3 of the next subsection, here are a few examples to get you started (feel free to come up with your own):

\begin{itemize}
	\item[-] \ldq It is not enough to be a good player, you must also play well." - Siegbert Tarrasch, chess master.
	\item[-] \ldq Big ideas, small details." - Michael Arkin, computer science teacher.
	\item[-] \ldq Nothing less."
	\item[-] \ldq It's not a bug, it's a feature!"
	\item[-] S.W.O.T.: \textbf{S}trengths, \textbf{W}eaknesses, \textbf{O}pportunities, \textbf{T}hreats
	\item[-] R.I.C.E.: \textbf{R}espect, \textbf{I}nnovation, \textbf{C}ooperation, \textbf{E}xcellence (courtesy of Satvik Bajaj).
	\item[-] A.B.C.: \textbf{A}lways, \textbf{B}e, \textbf{C}losing
\end{itemize}

Putting these to use as justification for a particular decision, or what have you, are an \textit{excellent} way to demonstrate creativity to your judge. Having one that is unique will, if done properly, make you stand out. For more ideas, see: \href{http://bighow.com/news/100-useful-acronyms-that-teach-us-about-writing-creativity-and-problem-solving-self-improvement-communication-and-more}{Creative Acronyms} and \href{http://www.hongkiat.com/blog/77-catchy-and-creative-slogans/}{Catchy and Creative Slogans}. Or you can read some of the millions of alternatives that a quick Google search yields. \\

It is also possible to experiment with the medium through which you present your case. Creating a visual graphic, e.g., a \textit{trifold} is a great way to engage the judge. It will also, if done properly, facilitate good structure and demonstrate impressive creativity. \\

Finally, if you really want to go the extra mile, learn about the current industrial climate. Talk to a \textit{real} finance professional when they have the time. This could potentially teach you what books and videos, and even practice, might not: experience and current knowledge.

\subsection{The Four I's}

The \lq Four I's' is a technique that, while does not follow the order of P.I.'s, may provide a very naturally flowing presentation. It 

\begin{enumerate}
	\item \textbf{Introduce:} here, it's good to include phrases such as "Hello! I'm very excited to be here today to \ldots ", or "Thank you for meeting with me! I'm grateful to be here today!". Make sure to be polite and \textbf{ask to be seated.} Sit on the edge of your seat! It is a very good idea to "\textit{lay out your road trip}", and detail what it is you're going to talk about. This provides a layout of the structure and ordering of your discussion with the judge.

	\item \textbf{Invert:} this section contains the bulk of your content in order of increasing specificity. The goal here is to start with the broader P.I.'s, such as \lq Explain the need to save and invest' and work towards something like \lq Explain types of investment'. Ensure that you include definitions, examples, analogies, and other unique content. The judge will like it if you present the case (that they've seen from many students already) in an interesting way. 
	
	\item \textbf{Impress:} go above and beyond here. This is where you may make some sort of judgement or plan of action related to the case (this stage might not even apply to your case). Discuss the implications of your decision and give a rationale and support for why you chose to do what you did. Consider using \textbf{B.E.T.}: \textbf{B}udget, \textbf{E}ffectiveness management, \textbf{T}imeline. Discuss what you're spending money on (in percentages, \textbf{not} hard numbers, such as $\$1400.00$), describe how you can measure whether your plan is working, and provide a timeline of how and when you will implement your major steps. Including this will give a capstone to your presentation and help convince the judge that you know what you are doing.
	
	\item \textbf{In conclusion:} give a summary of the things you talked about, in particular, highlight your discussion of the content of the P.I.'s. Be sure to ask the judge if they have any questions, and conclude the presentation by saying something along the lines of "If you have any questions, feel free to email me or give me a call" and thanking the judge.
	
\end{enumerate}

\begin{center}
 Good luck.
\end{center}
\end{document}

